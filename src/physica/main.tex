%------- ------- ------- Document Headers
\documentclass[10pt]{article} % latexmk, pdflatex
\usepackage[includefoot, paperheight=297mm, paperwidth=210mm, textwidth=170mm, textheight=257mm]{geometry} % texlive-geometry
\usepackage{lastpage} % texlive-lastpage
\usepackage{fancyhdr} % texlive-fancyhdr
\usepackage{graphicx} % 
\usepackage{indentfirst} % 
\usepackage{longtable} % 
\usepackage{array} % 
\usepackage{multicol} % 
\usepackage{multirow} % texlive-multirow
\usepackage{enumitem} % texlive-enumitem
\usepackage{mathtools} % texlive-mathtools
\usepackage{amsfonts} % texlive-amsfonts
\usepackage{amssymb} % 
\usepackage{hyperref} % texlive-hyperref
\usepackage{titlesec} % texlive-titlesec
\usepackage{tikz} % 
\usepackage{tkz-euclide} % texlive-tkz-euclide
%------- ------- ------- Initialization
\setlength{\parindent}{32pt}
\renewcommand{\baselinestretch}{0.7}
\renewcommand{\contentsname}{Sumário}
%------- ------- fancyhdr
\pagestyle{fancy}
\fancyhf{}
\renewcommand{\headrulewidth}{0pt}
\fancyfoot[R]{\thepage \hspace{1pt} / \pageref*{LastPage}}
%------- ------- array
\renewcommand{\arraystretch}{2}
%------- ------- multicol
\setlength{\columnsep}{32pt}
%------- ------- enumerate
\setlist{noitemsep, itemindent=64pt, labelwidth=0pt}
%------- ------- hyperref
\hypersetup{colorlinks=true, linkcolor=blue, pdfpagemode=FullScreen, pdftitle={Annotationes Meae}, pdfauthor={Gustavo Aragão}}
%------- ------- titlesec
\titleformat{\section}[block]{\normalfont\Large\bfseries}{\thesection}{16pt}{}
\titleformat{\subsection}[block]{\normalfont\large\bfseries}{\hspace{16pt}\thesubsection}{16pt}{}
\titleformat{\subsubsection}[block]{\normalfont\bfseries}{\hspace{32pt}\thesubsubsection}{16pt}{}
%------- ------- tikz
\usetikzlibrary{matrix}
\pgfdeclarelayer{bg}
\pgfdeclarelayer{fg}
\pgfsetlayers{bg,main,fg}
%------- ------- Commands
\newcommand{\eg}{\textit{e.g.:}}
%------- ------- ------- Beginning of Document
\begin{document}
    %------- ------- Title
    \title{
        \begin{tikzpicture}[scale=0.3, line join=bevel]
            % \a and \b are two macros defining characteristic
            % dimensions of the Penrose triangle.		
            \pgfmathsetmacro{\a}{2.5}
            \pgfmathsetmacro{\b}{0.9}
    
            \tikzset{
              apply style/.code     = {\tikzset{##1}},
              triangle_edges/.style = {thick,draw=black}
            }
    
            \foreach \theta/\facestyle in {
                0/{triangle_edges, fill = gray!50},
                120/{triangle_edges, fill = gray!25},
                240/{triangle_edges, fill = gray!90}
            }{
                \begin{scope}[rotate=\theta]
                    \draw[apply style/.expand once=\facestyle]
                    ({-sqrt(3)/2*\a},{-0.5*\a})                     --
                    ++(-\b,0)                                       --
                    ({0.5*\b},{\a+3*sqrt(3)/2*\b})                -- % higher point	
                    ({sqrt(3)/2*\a+2.5*\b},{-.5*\a-sqrt(3)/2*\b}) -- % rightmost point
                    ++({-.5*\b},-{sqrt(3)/2*\b})                    -- % lower point
                    ({0.5*\b},{\a+sqrt(3)/2*\b})                  --
                    cycle;
                \end{scope}
            }	
        \end{tikzpicture} \\
        \vspace{8pt} Annotationes Meae \\ Physica
    }
    \author{\textit{Gustavo Aragão}}
    \date{}
    \maketitle
    \thispagestyle{fancy}
    %------- ------- Table of Contents
    \tableofcontents
    \newpage
    %------- ------- 1
    \section{}
    %------- 1.1
\subsection{Conectivos}
    \begin{center}
        \begin{longtable}{| m{3cm} | m{12cm} |}
                \hline $ \neg $ & Negação\\
                \hline $ \vee $ & Disjunção\\
                \hline $ \wedge $ & Conjunção\\
                \hline $ \rightarrow $ & Implicação; Condicional\\
                \hline $ \leftrightarrow $ & Bi-implicação; Equivalência\\
                \hline
            \end{longtable}
    \end{center}
%------- 1.2
\subsection{Quantificadores}
    \begin{center}
        \begin{longtable}{| m{3cm} | m{12cm} |}
            \hline $ \forall $ & Todo; Qualquer\\
            \hline $ \exists $ & Existe; Pelo menos um; Algum\\
            \hline $ \exists ! $ & Existe exatamente um\\
            \hline
        \end{longtable}
    \end{center}
%------- 1.3
\subsection{Pontuações}
    \begin{center}
        \begin{longtable}{| m{3cm} | m{12cm} |}
            \hline $ \pm $ & Mais ou menos\\
            \hline $ = $ & Igual\\
            \hline $ \neq $ & Diferente\\
            \hline $ \equiv $ & Equivalente; Congruente\\
            \hline $ \sim $ & Aproximadamente\\
            \hline $ \approx $ & Aproximadamente igual\\
            \hline $ > $ & Maior que\\
            \hline $ \geq $ & Igual ou maior que\\
            \hline $ < $ & Menor que\\
            \hline $ \leq $ & Igual ou menor que\\
            \hline $ \star $ & Igual, maior ou menor que\\
            \hline $ \prec $ & Sequencialmente precede\\
            \hline $ \succ $ & Sequencialmente sucede\\
            \hline $ \propto $ & Proporcional\\
            \hline $ | $ & Tal que\\
            \hline $ ( ), [ ], \{ \} $ & Delimitadores\\
            \hline $ (\cdots)_p $ & Enupla de $p$ elementos\\
            \hline $ \{\cdots\} $ & Conjunto\\
            \hline $ \varnothing, \{\} $ & Conjunto vazio\\
            \hline $ \# $ & Número de elementos contidos\\
            \hline $ \nexists $ & Não existe\\
            \hline $ \ni, \supset $ & Contém\\
            \hline $ \not\ni, \not\supset $ & Não contém\\
            \hline $ \in, \subset $ & Está contido\\
            \hline $ \notin, \not\subset  $ & Não está contido\\
            \hline $ \supset $ & É subconjunto\\
            \hline $ \subset $ & É superconjunto\\
            \hline $ \cup $ & União\\
            \hline $ \cap $ & Interseção\\
            \hline $ \mathcal{P} $ & Propriedade; Lei de aplicação\\
            \hline $ \mathcal{D} $ & Domínio\\
            \hline $ \mathcal{CD} $ & Contradomínio\\
            \hline $ \mathcal{IM} $ & Imagem\\
            \hline $ \mathcal{C}^{B}_{A} $ & Complementar de $B$ em $A$\\
            \hline $ A^* $ & Conjunto dos elementos não nulos de $A$\\
            \hline $ A^n $ & Conjunto dos $n$ primeiros elementos de $A$\\
            \hline $ A_+ $ & Conjunto dos elementos positivos de $A$\\
            \hline $ A_- $ & Conjunto dos elementos negatios de $A$\\
            \hline $ \mathbb{N} $ & Conjunto dos números naturais\\
            \hline $ \mathbb{Z} $ & Conjunto dos números inteiros\\
            \hline $ \mathbb{Q} $ & Conjunto dos números racionais\\
            \hline $ \mathbb{I} $ & Conjunto dos números irracionais\\
            \hline $ \mathbb{R} $ & Conjunto dos números reais\\
            \hline $ \mathfrak{I} $ & Conjunto dos números imaginários\\
            \hline $ \mathbb{C} $ & Conjunto dos números complexos\\
            \hline $ \mathbb{U} $ & Conjunto universo\\
            \hline $ \mapsto $ & Função aplicada\\
            \hline $ \mathrm{P}(n) $ & Polinômio de Grau $n$\\
            \hline $ \therefore $ & Portanto\\
            \hline $ \because $ & Em razão de\\
            \hline $ \infty $ & Infinito\\
            \hline $ \Delta $ & Variação\\
            \hline $ \lim\limits_{a \to b} f(x) $ & Limite\\
            \hline $ f', \frac{dy}{dx} $ & Derivada\\
            \hline $ \frac{\partial y}{\partial x}  $ & Derivada parcial\\
            \hline $ \int_a^b $ & Integral\\
            \hline $ \oint_a^b $ & Integral de linha\\
            \hline $ \nabla $ & Gradiente\\
            \hline $ \displaystyle\sum_{i=a}^{b} f(i) $ & Somatório\\
            \hline $ \displaystyle\prod_{i=a}^{b} f(i) $ & Produtório\\
            \hline $ \mathring{APB} $ & Ponto $P$ contido entre $A$ e $B$\\
            \hline $ \overline{AB} $ & Segmento de reta\\
            \hline $ \overrightarrow{AB} $ & Semirreta\\
            \hline $ \hat{\alpha}, \hat{AOB} $ & Ângulo\\
            \hline $ \parallel $ & Paralelo\\
            \hline $ \nparallel $ & Concorrente\\
            \hline $ \bot $ & Perpendicular\\
            \hline $ \triangle ABC $ & Triângulo ABC\\
            \hline $ \Diamond ABCDE $ & Polígono ABCDE\\
            \hline
        \end{longtable}
    \end{center}
%------- 1.4
\subsection{Alfabeto Grego}
    \begin{center}
        \begin{longtable}{| m{3cm} | m{3cm} | m{8.57cm} |}
            \hline $ \alpha $ & $ A $ & Alfa\\
            \hline $ \beta $ & $ B $ & Beta\\
            \hline $ \gamma $ & $ \Gamma $ & Gama\\
            \hline $ \delta $ & $ \Delta $ & Delta\\
            \hline $ \epsilon $ & $ E $ & Epsilon\\
            \hline $ \zeta $ & $ Z $ & Zeta\\
            \hline $ \eta $ & $ H $ & Eta\\
            \hline $ \theta $ & $ \Theta $ & Theta\\
            \hline $ \iota $ & $ I $ & Iota\\
            \hline $ \kappa $ & $ K $ & Kappa\\
            \hline $ \lambda $ & $ \Lambda $ & Lambda\\
            \hline $ \mu $ & $ M $ & Mu\\
            \hline $ \nu $ & $ N $ & Nu\\
            \hline $ \xi $ & $ \Xi $ & Xi\\
            \hline $ o $ & $ O $ & Omicron\\
            \hline $ \pi $ & $ \Pi $ & Pi\\
            \hline $ \rho $ & $ P $ & Rho\\
            \hline $ \sigma $ & $ \Sigma $ & Sigma\\
            \hline $ \tau $ & $ T $ & Tau\\
            \hline $ \upsilon $ & $ \Upsilon $ & Upsilon\\
            \hline $ \phi $ & $ \Phi $ & Phi\\
            \hline $ \chi $ & $ X $ & Chi\\
            \hline $ \psi $ & $ \Psi $ & Psi\\
            \hline $ \omega $ & $ \Omega $ & Omega\\
            \hline
        \end{longtable}
    \end{center}
%------- ------- ------- End of Document
\end{document}